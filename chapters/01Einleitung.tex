\chapter{Einleitung}
\todo{Muss noch gemacht werden. Zum Schluss}
\section{Motivation}
Bei der Betreuung von Patienten durch einen Psychotherapeuten sind Hausaufgaben oft Bestandteil der Behandlung. Hierbei ist es für den Therapeuten jedoch schwierig, zu beurteilen wie effektiv diese Hausaufgaben wirklich sind. Schon aus dem Grund, dass er nur das Feedback durch den Patienten hat. Oft werden die Übungen auch vergessen oder schlicht nicht gemacht. Was eine Qualitative Auswertung über den nutzen der Hausaufgaben unmöglich macht. Im Zuge dieser Arbeit wurde das Konzept einer Anwendung entwickelt, die auf der einen Seite dem Therapeuten Daten und Feedback über die Aufgaben gibt. Auf der anderen Seite soll sie aber auch dem Patienten dabei helfen, seine Aufgabe zu erledigen. Hierbei wäre es auch Denkbar, den Patienten spielerisch dazu zu bringen seine Hausaufgaben zu machen. Vorstellbar ist auch ein Erfolgssystem oder Anbindung an die Krankenkasse. Durch derartige Mechanismen könnte der Patient weiter Motiviert werden und die Behandlung dadurch verbessert.

\section{Aufbau der Arbeit}
\todo{Muss noch gemacht werden. Später}