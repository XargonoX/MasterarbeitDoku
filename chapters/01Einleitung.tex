\chapter{Einleitung}

\section{Motivation}
Um ein grundlegendes Verständnis für eine Regelung in einem eingebetteten System zu erlangen, soll im Rahmen des Labors Eingebettete Systeme eine Regelung entworfen werden, die die Drehzahl eines Lüfters auf einen gegebenen Wert konstant halten soll. Hierfür wurden zu Beginn die Grundlagen für eine Implementierung erarbeitet und Teilkomponenten erstellt, welche im späteren Regler erneut verwendet werden. Als Leitfaden und Aufgabenstellung wurde das Skript \cite{LES16} bereitgestellt.

\section{Aufbau der Arbeit}
Diese Ausarbeitung soll den Ablauf vorstellen, in welchem der Regler implementiert wurde. Im Kapitel~\ref{Theoretische Grundlagen} werden die theoretischen Grundlagen eines Regelkreises kurz umrissen. Die VHDL-Implementierung und die dazugehörige Software wird im Kapitel~\ref{Implementierung} vorgestellt. Insbesondere soll auf die Entscheidung eingegangen werden, welche Teilkonzepte in Hardware und welche in Software realisiert wurden. Abschließend wurde die Funktionsweise des Reglers mittels Tests gezeigt, welche im Kapitel~\ref{Tests} beschrieben sind.