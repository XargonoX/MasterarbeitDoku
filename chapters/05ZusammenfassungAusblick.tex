\chapter{Zusammenfassung und Ausblick} \label{_ZusammenfassungundAusblick}
Im Anschluss an die Theorie wird im folgenden Kapitel eine Zusammenfassung der Arbeit gegeben. An Anschluss daran wird ein Ausblick auf eine mögliche Weiterentwicklung der im Rahmen dieser Arbeit konzipierten und realisierten Plattform gegeben. 
\section{Zusammenfassung}
Im Verlauf dieser Arbeit wurde eine Plattform zur Unterstützung von Therapeutischen Betreuungen konzipiert und realisiert. Diese kann in jedweder Therapeuten - Patienten Beziehung eingesetzt werden und soll den, für den Therapieerfolg, so wichtigen Transfer der Therapie in den Alltag erleichtern.

Der Fokus der Plattform liegt momentan zum einen auf den Aufgaben und Übungen welche der Patient Zuhause selbstständig erledigen soll und zum anderen auf Fragebögen welche vom Therapeut erstellt und vom Patient ausgefüllt werden können.

Zum einen soll mit dieser Plattform der Patient unterstützt werden indem ihm dabei geholfen wird seine Aufgabe/Übung wie vorgegeben zu erledigen. Da ein großes Problem von derartigen Therapien ist, dass die Aufgaben/Übungen nicht, oder nicht korrekt ausgeführt werden.

Zum anderen soll aber auch der Therapeut Informationen und Daten geliefert bekommen. Dieses Feedback kann wiederum dazu eingesetzt werden die Therapie sowie die gestellten Aufgaben/Übungen besser an den Patienten anzupassen. Des weiteren können die erhobenen Daten in der Wissenschaft verwendet werden um die Therapie Grundlegend zu verbessern.

Aus der Software Analyse ging hervor, dass ein geeigneter Weg zur Realisierung einer solchen Plattform ein Client für den Therapeuten in Form einer Webanwendung so wie einer Cross-Plattform-App für Android und iOS als Patienten Client ist. Der Daten Austausch dieser beiden Programme erfolgt über einen im Internet erreichbaren Server, welcher die Daten beider Seiten persistent in einer Datenbank speichert.

Die Webanwendung dient dem Therapeuten dazu die Patienten, Aufgaben-/Übungsvorlagen sowie Fragebögen zu verwalten. Diese können anschließend den jeweiligen Patienten zu bestimmten Zeiten und Wochentagen zugewiesen werden. An den so spezifizierten Zeitintervallen wird der Patient am Anfang des Zeitintervalls daran erinnert, dass er seine Aufgabe zu erledigen hat. Am Ende des Zeitintervalls hat er die Möglichkeit den von Therapeut angehängten Fragebogen auszufüllen.

Innerhalb der App muss der Patient zuerst Verbindung zu dem vom Therapeuten angelegten Profil ausnehmen. Anschließend hat er eine Übersicht über die ihm zugeteilten Aufgaben/Übungen. Über diese Ansicht gelangt der Nutzer auch zur Detailansicht einer Aufgabe in welcher deren Beschreibung und angehängte Materialien gezeigt werden.

Die meisten der Anforderungen welche in der Anforderungsanalyse \ref{Anforderungsanalyse} spezifiziert wurde, konnten im Zuge dieser Arbeit in das Konzept eingearbeitet werden. Aus dem Anforderungsabgleich \ref{Anforderungsabgleich} geht hervor, dass abgesehen vom Sicherheitsaspekt eine lauffähige Software entstanden ist. Was wiederum zeigt, dass es Möglich ist, die Therapeut - Patient Beziehung auf einem Zeitgemäßen Weg zu unterstützen. Auf eine Weise in welcher beide Seiten gleichermaßen dazu beitragen den Therapieerfolg zu erhöhen.  

\section{Ausblick}
Der nächste Schritt bei der Weiterentwicklung dieser Plattform sollte in erster Linie einmal der Sicherheitsaspekt der kommunizierten Daten sein, noch bevor diese mit realen Patientendaten verwendet wird. Anschließend sollten die sonstigen Anforderungen welche im Anforderungsabgleich \ref{Anforderungsabgleich} als \textit{nicht implementiert} markiert sind in die Plattform eingepflegt werden.

Hierzu gehören vor allem die Kontrollmechanismen da es für den Therapeuten oft schwierig ist, zu beurteilen ob eine Aufgabe/Übung wirklich gemacht wurde. Aus dem Grund, dass er nur das Feedback durch den Patienten hat, welche oft zum Lügen neigt. Dieser Umstand macht es dem Therapeuten schwer, zu beurteilen wie effektiv die gestellten Aufgaben wirklich sind. 

Um die Aufgaben- und Übungsvorlagen global zu verbessern, sollte der Plattform eine Feature hinzugefügt werden, welches es erlaubt die gesammelten Daten im größeren Stil weiter zu verarbeiten. Momentan können die von den Patienten erzeugen Daten über die vom Datenbank Server bereitgestellte API abgerufen werden und liegen dann für die Weiterverarbeitung im JSON-Format zu Verfügung.

Hierzu wäre es Denkbar die Plattform auch zur Interaktion zwischen Therapeuten untereinander oder mit der Forschung zu verwenden. Es könnten sowohl die Aufgaben/Übungsvorlagen als auch die Fragebögen in einer Art Marktplatz angeboten werden. Dies setzt jedoch voraus, dass es dem Therapeuten möglich gemacht wird, Materialien auf die Plattform hochzuladen und somit individuellere Aufgaben-/Übungsvorlagen zu erstellen. Hierbei sollte nicht außer Acht gelassen werden, dass der Client des Patienten allerhand Peripherie und Sensoren bietet welche in die Gestaltung der Aufgaben und Übungen mit einfließen können sollten.  

Des weiteren wäre es Denkbar, den Client des Therapeuten so zu erweitern, dass er in diesem seine komplette Verwaltung der Patienten erledigen kann. Andernfalls wäre es in den meisten Praxen wohl nötig die Patienten innerhalb dieser Plattform zusätzlich zu verwalten. 

Im Fall, dass sich diese Plattform wirklich über Praxis und Forschung ausdehnt, wäre es auch durchaus Denkbar, dass sich ein \textit{sich krankfühlender Mensch} den Client für den Patienten herunter lädt, seine Leiden schildert und noch bevor er den ersten Termin bei seinem Therapeuten in der Nähe hat schon Hilfe bekommt. In erster Linie voll Automatisch bis hin zur Verbindung mit dem Therapeuten seiner Wahl.

Neben der Weiterentwicklung der Plattform, ist es dringend Notwendig den Mehrwert dieser zu untersuchen. Im Zuge einer Studie sollte untersucht werden, ob dem Patienten durch seinen Client dabei geholfen wird,seine Aufgabe/Übung zu erledigen und ob die engere Bindung zwischen Therapeut und Patient zu einem besseren Therapieergebnis führt.
Eine weitere Frage ist, in wie weit ein Patient bereit wäre seine erzeugten Daten mit der Wissenschaft zu Teilen. Hierbei fällt auch die von den Krankenkassen oft geforderte, bessere Einsicht in die Patientendaten ein. Durch Vergünstigungen beim Krankenkassenbeitrag könnte der Patient weiter Motiviert werden seine Aufgaben zu erledigen. Hierbei wäre es auch Denkbar, den Patienten spielerisch dazu zu bringen seine Aufgaben und Übungen zu machen oder dem Therapeuten das gewünschte Feedback zu bringen.
