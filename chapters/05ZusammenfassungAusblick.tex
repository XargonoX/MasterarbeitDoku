\chapter{Zusammenfassung und Ausblick} \label{_ZusammenfassungundAusblick}
Im Anschluss an die Theorie soll nun eine Zusammenfassung der Arbeit so wie ein Ausblick auf eine mögliche Weiterentwicklung der im Rahmen dieser Arbeit konzipierten und realisierten Plattform gegeben werden. 
\section{Zusammenfassung}
Im Verlauf dieser Arbeit wurde eine Plattform zur Unterstützung von Therapeutischen Betreuungen konzipiert und realisiert. Diese kann in jedweder Therapeuten - Patienten Beziehung eingesetzt werden und soll den, für den Therapieerfolg, so wichtigen Transfer der Therapie in den Alltag erleichtern.

Der Fokus der Plattform liegt momentan zum einen auf den Aufgaben und Übungen welche der Patient Zuhause selbstständig erledigen soll und zum anderen auf Fragebögen welche vom Therapeut erstellt und vom Patient ausgefüllt werden können.

Zum einen soll mit dieser Plattform der Patient unterstützt werden indem ihm dabei geholfen wird seine Aufgabe/Übung wie vorgegeben zu erledigen. Da ein großes Problem von derartigen Therapien ist, dass die Aufgaben/Übungen nicht, oder nicht korrekt ausgeführt werden.

Zum anderen soll aber auch der Therapeut Informationen und Daten geliefert bekommen. Dieses Feedback kann wiederum dazu eingesetzt werden die Therapie sowie die gestellten Aufgaben/Übungen besser an den Patienten anzupassen. Des weiteren können die erhobenen Daten in der Wissenschaft verwendet werden um die Therapie Grundlegend zu verbessern.

Die somit entstandene Plattform bietet einen Client für den Therapeuten in Form einer Internetseite so wie einer Cross-Plattform-App für Android und iOS als Patienten Client. Der Daten Austausch dieser beiden Programme erfolgt über einen im Internet erreichbaren Server welcher die Daten beider Seiten persistent speichert.

 
\section{Ausblick}
Gesammelte Daten auswerten.
Patientenverwaltung ausdehnen auf Patientenverwaltungstool.
Kontrollmechanismen.
Hochladen von Materialien.
Andere Materialien...
Erweiterung auf internet Doc.



---
 Hierbei ist es für den Therapeuten jedoch schwierig, zu beurteilen wie effektiv die Übungen wirklich sind. Schon aus dem Grund, dass er nur das Feedback durch den Patienten hat. Oft werden die Übungen auch vergessen oder schlicht nicht gemacht. Was eine Qualitative Auswertung über den nutzen der Hausaufgaben unmöglich macht.  Hierbei wäre es auch Denkbar, den Patienten spielerisch dazu zu bringen seine Hausaufgaben zu machen. Vorstellbar ist auch ein Erfolgssystem oder Anbindung an die Krankenkasse. Durch derartige Mechanismen könnte der Patient weiter Motiviert werden und die Behandlung dadurch verbessert. 
 ---