\chapter{Anforderungsabgleich} \label{Anforderungsabgleich} \todo{Anforderungstabellen mit erwähnen, wo kommen die Anforderungen her}
Im folgenden Kapitel wird ein Anforderungsabgleich vorgenommen, dieser gibt Aufschluss darüber welcher der in Kapitel Anforderungsanalyse \ref{Anforderungsanalyse} spezifizierten Anforderungen schlussendlich implementiert wurden.
Hierbei wird im einzelnen auf die jeweiligen Anforderungen eingegangen und Querverweise in das Kapitel Implementierung \ref{_Implementierung} hergestellt. Um die Reihenfolge von Kapitel \ref{Anforderungsanalyse} beizubehalten wird mit den Funktionalen Anforderungen begonnen, anschließend werden die Nicht-Funktionalen Anforderungen untersucht.
\section{Funktionale Anforderungen}
In diesem Kapitel werden die funktionalen Anforderungen untersucht. Sowohl an den Datenbank Server sowie für die beiden Clients.
\subsection{Datenbank Server}
Im folgenden Abschnitt werden die Anforderungen an den Datenbank Server aus Tabelle \ref{NichtFunktionaleAnforderungenServer} untersucht.
\paragraph{Anforderung 1.1 - Datenpersistenz  \textcolor{green}{\checkmark}}
 wurde wie in Kapitel \ref{_ImpDatenbankServer} beschreiben umgesetzt. Für die Datenpersistenz wurde eine MongoDB Datenbank auf dem Server installiert, diese legt ihr Daten für den Nutzer nicht sichtbar auf dem Dateisystem ab, wo diese persistent gespeichert werden.

\paragraph{Anforderung 1.2 - Daten Sicherung  \textcolor{red}{X}}
Dieses Feature wurde nicht Implementiert. Da bisher nicht mit realen Daten bearbeitet wurde war auch eine Sicherung dieser nicht nötig.

\paragraph{Anforderung 1.3 - Datenschnittstelle \textcolor{green}{\checkmark}}
Wie in Kapitel \ref{_ImpDatenbankServer} beschreiben, wurde mit Hilfe des express Frameworks eine REST-API implementiert welche HTTP anfragen entgegen nimmt. Hierdurch ist es Möglich, Daten aus der Datenbank anzufragen, zu verändern bzw. löschen so wie neue Datensätze anzulegen.


\subsection{Therapeuten Client}
Die Anforderungen an den Therapeuten Client \textbf{wurden alle erfüllt.}
\paragraph{Anforderung 2.1 - Patientenverwaltung \textcolor{green}{\checkmark}}
Dieses Feature \textbf{wurde Implementiert} indem dem Therapeuten Client wie in Kapitel \ref{_ImpTCPatientUebersicht} beschreiben, eine Seite hinzugefügt wurde auf derer er eine Übersicht über seine Patienten hat. Hier können Patienten auch gelöscht werden - Anforderung 2.1.3.

Zusätzlich wurde wie in Kapitel \ref{_ImpTCPatientDetail} beschreiben eine Seite erzeugt mit welche der Therapeut Patienten anlegen bzw. editieren kann - Anforderungen 2.1.1 und 2.1.2.

\paragraph{Anforderung 2.2 - Aufgabenverwaltung \textcolor{green}{\checkmark}}
Mittels der in Kapitel \ref{_ImpTCAufgaben} beschriebenen Seiten ist es dem Therapeuten möglich die Vorlagen für Aufgaben / Übungen zu verwalten. Wie bei der Patientenverwaltung gibt es eine Übersicht in welcher einzelne Aufgaben ausgewählt oder gelöscht (Anforderung 2.2.3) werden können. Mittels der ebenso in Kapitel \ref{_ImpTCAufgaben} beschriebenen Seite zum anlegen und editieren von Aufgaben werden den Anforderungen 2.2.1 und 2.2.2 abgedeckt.

\paragraph{Anforderung 2.3 - Fragebogenverwaltung \textcolor{green}{\checkmark}}
Wie für die Verwaltung von Patienten und Aufgaben/Übungen wurde auch für die Fragebogenverwaltung eine eigene Seite Implementiert, welche in Kapitel \ref{_ImpTCFragebogen} beschreiben wird. Diese bietet die Möglichkeit Fragebögen zu löschen(Anforderung 2.3.3). Die ebenfalls in Kapitel \ref{_ImpTCFragebogen} beschriebene Seite zum erstellen und editieren von Fragebögen erfüllt die Anforderungen 2.3.1 und 2.3.2.

\subsection{Aufgaben-/Übungsvorlagen}
Die Anforderungen an die Aufgaben-/Übungsvorlagen wurden, wie in Tabelle \ref{TabelleAnforderungsabgleichVorlagen} beschrieben, Teilweise erfüllt.
A.1 bis A.3 sowie A.6 wurden wie in Kapitel \ref{_ImpTCAufgaben} beschreiben erfüllt. 

\begin{table}[htbp]
	\begin{center}
		\begin{tabular}{p{0,5cm} p{6,5cm} p{7,5cm}}
			\rowcolor{black!20} \textbf{Nr.} & \textbf{Bezeichnung} & \textbf{Erfüllt?} \\ \toprule 
			A.1 & \textbf{Name} & \textcolor{green}{\checkmark} \\ \hline \addlinespace
			A.2 & \textbf{Beschreibung} & \textcolor{green}{\checkmark} \\ \hline \addlinespace
			A.3 & \textbf{Materialien} & \textcolor{green}{\checkmark} \\ \hline \addlinespace
			A.4 & \textbf{Kontrollmechanismen} & \textcolor{red}{X} \\ \hline \addlinespace
			A.5 & \textbf{Kategorisierung} & \textcolor{red}{X}  \\ \hline \addlinespace
			A.6 & \textbf{Verschiedene Erledigungskontexte} & Es wurde nur der Zeitkontext implementiert \\ \hline \addlinespace
		\end{tabular}
	\end{center}
	\caption[Anforderungsabgleich der Aufgaben-/Übungsvorlagen]{Anforderungsabgleich der Aufgaben-/Übungsvorlagen}
	\label{TabelleAnforderungsabgleichVorlagen}
\end{table}

\subsection{Fragebögen}
Die in Tabelle \ref{TabelleFunktionaleAnforderungenFragebogen} gezeigten Anforderungen wurden abgesehen von Anforderung B.2 \textbf{erfüllt}. Anforderung B.2 wurde als nicht mehr wichtig erachtet, da der Name ausreicht, um einen Fragebogen zu identifizieren. Durch die grafische Darstellen mittels des in Kapitel \ref{_ImpTCFragebogen} beschrieben Tools zur Erstellung/Bearbeitung von Fragebögen wird das wiederfinden eines bestimmten Fragebogens weiter vereinfacht.

\textbf{Anforderung B.3} wurde erfüllt, indem es mittels des grafischen Tools möglich ist, die Fragen und Antworten nach belieben mit einander zu verbinden. Hierdurch kann der Therapeut frei wählen, welche Frage auf eine gegebene Antwort folgt.

Verschiedene Antwortmöglichkeiten wie in \textbf{Anforderung B.4} gefordert wurden wie in Kapitel \ref{_ImpTCFragebogen} beschreiben mit Hilfe eines Zusatz Panels realisiert. Da dies aber nicht sehr intuitiv ist, sollte hier die Möglichkeit wahrgenommen werden, dass man in dem verwendeten Tool neue Elemente einbauen kann. Hierdurch wäre es Beispielsweise möglich zusätzliche Elemente für die einzelnen Fragetypen zu Verfügung zu stellen. Da in dem Tool auch auf Klicks reagiert werden kann, wäre zur Vereinfachung dieses Vorgangs auch ein Kontextmenü denkbar, in welchem man einem allgemeinen Frageelement einen Typ zuweisen kann.

\paragraph{Fragetypen}
Wie in der Tabelle \ref{FrageTypen} im Kapitel \ref{_ImpTCFragebogen} zu sehen wurden die Anforderungen C.1 bis C.4 in vier verschiedenen Typen von Fragen realisiert.


\begin{description}
	\item[Anforderung C.1 (Einzelantwort)]\hfill \\
	 wurde als Typ \textbf{single} spezifiziert.
	\item[Anforderung C.2 (Mehrfachantwort)] \hfill \\
	 wurde als Typ \textbf{multi} spezifiziert.
	\item[Anforderung C.3 (Bewertung)]\hfill \\
	wurde als Typ \textbf{text} bewertet.
	\item[Anforderung C.4 (Freitext)]\hfill \\
	wurde als Typ \textbf{rating} spezifiziert.
\end{description}

\subsection{Patienten Client}
Die Anforderungen 3.1 bis 3.7 aus Tabelle \ref{TabelleFunktionaleAnforderungenTherapeutClient}  wurden wie in Kapitel \ref{_ImpPatientClient} beschrieben implementiert. Ein Überblick über die einzelnen Anforderungen mit Querverweisen in das Implementierungskapitel sind in Tabelle \ref{TabelleAnforderungsabgleichPatientClient} zu finden.

\begin{table}[H]
	\begin{center}
		\begin{tabular}{p{0,5cm} p{4cm} p{2cm} p{2,5cm}}
			\rowcolor{black!20} \textbf{Nr.} & \textbf{Anforderung} & \textbf{Erfüllt?} & \textbf{Beschreibung} \\	\toprule
			3.1 & User Identifikation & \textcolor{green}{\checkmark} & \ref{_ImpPatientClient} \\ \hline \addlinespace
			3.2 & Aufgabenanzeige & \textcolor{green}{\checkmark} & \ref{_ImpPCAufgaben} \\ \hline \addlinespace
			3.3 & Detailanzeige der Aufgabe  & \textcolor{green}{\checkmark} & \ref{_ImpPCAufgabenDetail} \\ \hline \addlinespace
			3.4 &Anzeige der Materialien & \textcolor{green}{\checkmark} & \ref{_ImpPCAufgabenDetail} \\ \hline \addlinespace
			3.5 &Verändern des Erledigungskontext & \textcolor{green}{\checkmark} & \ref{_ImpPCAufgabenDetail} \\ \hline \addlinespace
			3.6 &Erinnerung & \textcolor{green}{\checkmark} & \ref{_ImpPCErinnerungen} \\ \hline \addlinespace
			3.7 &Fragebogen beantworten & \textcolor{green}{\checkmark} & \ref{_ImpPCFragebogen} \\ \hline \addlinespace
		\end{tabular}
	\end{center}
	\caption[Anforderungsabgleich des Patienten Clients]{Anforderungsabgleich des Patienten Clients}
	\label{TabelleAnforderungsabgleichPatientClient}
\end{table} 

\section{Nicht-funktionale Anforderungen}
Im folgen Kapitel werden die nicht-funktionalen Anforderungen beleuchtet. 
\subsection{Datenbank Server}
Die nicht-funktionalen-Anforderungen des Datenbank Servers welche in Tabelle \ref{NichtFunktionaleAnforderungenServer} aufgezeigt sind, werden im folgenden Abschnitt untersucht.
\paragraph{Anforderung 4.1 - kurze Reaktionszeiten  \textcolor{green}{\checkmark}} Da der Server auf erprobten Technologien wie NodeJS und dem Express Framework beruht. Kann davon aus gegangen werden das diese auch für größere Nutzerzahlen geeignet sind. In der erstellten Testumgebung ist die Reaktionszeit der Anwendung nie negativ aufgefallen.
\paragraph{Anforderung 4.2 - Sicherheit  \textcolor{red}{X}} Da die Plattform bisher nicht mit realen Patientendaten verwendet wurde, war keine Notwendigkeit diese zu verschlüsseln. Bevor die Plattform mit realen Daten verwendet wird, müssen diese vorher vor dem Zugriff unbefugter geschützt werden. Hierzu sind im Kapitel \ref{_GrundlagenSicherheit} Anregungen gegeben.

\subsection{Therapeuten / Patienten Client}
Im folgen Absatz werden die in Tabelle \ref{NichtFunktionaleAnforderungenClients} spezifizierten, nicht-funktionalen Anforderungen an die beiden Clients untersucht.
\paragraph{Anforderung 5.1 - Natives Design \textcolor{green}{\checkmark}} Da der \textbf{Therapeuten Client} in Form einer Website implementiert wurde, kann es für diesen kein Natives Design geben. Dieser hat dann aber den Vorteil, dass er auf jedem Gerät gleich aussieht. Einzig bei mobile Endgeräte, gemessen an der Displaygröße, wird das User-Interface der Anwendung angeglichen um kleinen Displays gerecht zu werden. Darauf, welches Betriebssystem zugrunde liegt wird nicht reagiert.

Das native Design des \textbf{Patienten Clients} wurde durch die Verwendung von Ionic weitgehendst umgesetzt. Die wichtigsten Bedienelemente werden durch Ionic an das jeweilige Betriebssystem angeglichen. Man sieht jedoch schon noch, dass die App im Herzen eine Website ist. Durch eine Überarbeitung des User-Interfaces kann das native Feeling der App extrem erhöht werden.

\paragraph{Anforderung 5.2 - vertretbare Reaktionszeiten \textcolor{green}{\checkmark}} Da für beide Clients sehr erprobte Technologien verwendet wurden und die Anwendungen auch keine großen Rechenleistungen benötigen wurde keine Untersuchungen zu den Reaktionszeiten vorgenommen. Während der Verwendung der Testumgebungen sind auch hier keine unzumutbaren Verzögerungen aufgefallen.

