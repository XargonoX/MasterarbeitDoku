\chapter{Theoretische Grundlagen} \label{Theoretische Grundlagen}
Im diesem Kapitel werden die Theoretischen Grundlagen der verwendeten Technologien im einzelnen betrachtet. Hierbei wird nach der "was, warum, wie"(eng. what, why, how) Methode vorgegangen. 
\section{(Web)Grundlagen}
REST, HTML(GET,SET,...),Client-Server Architektur, JavaScript
\section{Mögliche Frameworks / Multi Plattform}
Eine der Grundvoraussetzungen dieser Arbeit sollte die Verfügbarkeit auf möglichst vielen Plattformen sein. Auch wenn während dieser Arbeit aus verschiedenen Gründen, die später noch aufgezeigten werden, früh klar war das die Wahl auf den sogenannten MEAN-Stack fällt, werden hier dennoch einige andere Frameworks und Plattformen aufgezeigt. Sowohl für die Desktop Anwendung des Therapeuten, also auch für die mobile Anwendung des Patienten.
\subsection{Möglichkeiten der Anwendungsentwicklung}
\subsubsection{Plattformspezifische Entwicklung}
Was sind die Vor und nachteile von Plattformspezifischer entwicklung.
\subsubsection{Cross Plattform Entwicklung}
Von Cross Plattform Entwicklung spricht man, wenn der Code der entwickelten Anwendung nicht nur für eine spezifische Plattform verwendbar ist. Es gibt etliche verschiedenen technologischen Ansätze \todo{}

\subsubsection{Die Technologie: Nativ, Web oder Hybrid?}

durch verschiedene Technologien
\subsection{Desktop}
Angular, Qt, ...
\subsection{Mobile Anwendung}
Ionic(PhoneGap),Xamarin  

 (http://thinkapps.com/blog/development/develop-for-ios-v-android-cross-platform-tools/) 

